\documentclass[11pt]{article}
\usepackage{amssymb,amsmath}
\usepackage{dsfont}
\usepackage{times}
\usepackage[left=1in, right=1in, top=1in, bottom=0.5in, includefoot]{geometry}
\setlength\parindent{0.25in}
\setlength\parskip{1mm}

\title{CMSC601 Paper Summaries}
\author{Patrick Trinkle\\
Dept. of Computer Science and Electrical Engineering,\\
University of Maryland Baltimore County,\\
Baltimore, MD, 21250\\
\texttt{tri1@umbc.edu}}
\date{February 21st, 2010}

\begin{document}

\maketitle

\section{Parberry Summary}
Author: Ian Parberry\\
Title: A Guide for New Referees in Theoretical Computer Science

This paper makes an attempt to describe the refereeing process.  Refereeing is a major part of being an active scientist in academia.  The evidence provided to support this claim is mostly anecdotal, as well as citing a few papers previously written on the subject.  It is well organized and clearly written.  Good computer scientist community members referee papers.  The refereeing process is fairly straightforward provided you are fair, honest, and thorough.  Parberry describes the ethics for refereeing.  It's also important that referees who are fair and quick will have the favor returned when it is time for them to publish.  Referees weed out papers as well as provide feedback for improving a paper in terms of "correctness, significance, innovation, interest, timeliness, succinctness, accessibility, elegance, readability, style, and polish."\cite{Parberry1994}  While describing these things in detail as well as various other details of a paper a referee must watch for, Parberry incidentally provides a lot of good advice for researchers writing papers.  This includes not over using passive voice, and to consider that a paper may be boring and could use some reworking.

Parberry defines seven types of papers.  For nearly all types a specific example of a published scientific paper is cited.  This is used as evidence to that claim.  I don't feel this paper is a seminal paper, but it is very useful to scientists who are new to refereeing in any field.  It might be worth determining if the process as described over ten years ago is still the same.  Also, worth investigating is a paper cited by this one written about ten years prior, D. Gifford "How to referee a research paper"  1982,  to determine if the process evolved.

\section{Smith Summary}

Author: Alan Smith\\
Title: The Task of the Referee

Smith details the methods an evaluator uses to referee a paper for publishing.  This paper describes evaluating proposals as well.  Smith discusses the problem of what makes a paper publishable, as well as the roles of all individuals involved in the process.  This paper is not original in content, but is a good collection of information on the topic of refereeing and paper writing.  Smith does reference previous papers on refereeing.  The paper is well organized and clearly written.

A publishable paper is defined as one that makes a "sufficient contribution."\cite{ASmith} A referee's goal is to provide an opinion and also to support this opinion.  When refereeing a paper it is important to not have any presumptions about the paper as well as to not automatically dismiss a paper if it is contrary to popular wisdom.  Because it is in the best interest of the author for the report to be taken as helpful, the referee courteous and professional in all remarks.  Smith explains quite a few things to someone writing a research paper or proposal regarding ethics and decorum for submissions.  This is evidenced through general knowledge versus cited research.

A proposal is defined as more of an idea for future work.  This idea needs to be thought out but not necessarily be all in one proposal.  Because it is more likely to get a smaller amount of funding, it is sufficient to propose part of the work.  If the researcher reaches this goal, then they are more likely to receive follow-on money to continue working.  This reduces the risk on the investor.

\begin{thebibliography}{5}
\bibitem[Parberry1994]{Parberry1994}Ian Parberry, ``A Guide for New Referees in Theoretical Computer Science," 1994
\bibitem[ASmith]{ASmith}Alan Smith, ``The Task of the Referee,"
\end{thebibliography}

\end{document}
